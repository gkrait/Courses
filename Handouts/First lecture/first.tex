\documentclass{article}
\usepackage[utf8]{inputenc}





\newtheorem{thm}{Theorem}%[subsection]
\newtheorem{lem}[thm]{Lemma}
\newtheorem{cor}[thm]{Corollary}
\newtheorem{prop}[thm]{Proposition}
\newtheorem{conj}[thm]{Conjecture}
\newtheorem{dfn}[thm]{Definition}
\newtheorem{nota}[thm]{Notation}
\newtheorem{dfnnota}[thm]{Definition and Notation}
\newtheorem{notarem}[thm]{Definition and Remark}
\newtheorem{con}[thm]{thm}
\newtheorem{rem}[thm]{Remark}
\newtheorem{claim}[thm]{Claim}
\newtheorem{ex}[thm]{Example}
\newtheorem{prob}[thm]{Problem}
\newtheorem{ass}[thm]{Assumption}
\newtheorem{obs}[thm]{Observation}
\newtheorem{moti}[thm]{Motivation}
\newtheorem{cons}[thm]{Construction}
\newtheorem{setup}[thm]{Computational Setup}
\newtheorem{dfns}[thm]{Definitions}
\newtheorem{axiom}[thm]{Axiom}
\newtheorem{dis}[thm]{Discussion}




\title{sdas}
\author{georgekrait }
\date{September 2020}
\begin{document}

\maketitle


\section{Database in general}

\paragraph{Database:} A database is an organized collection of data,
stored and accessed electronically from a computer system where it is possible to interact with data. How?

 \paragraph{Database management system (DBMS):} software system that enables users to create, delete,
maintain and control access to the database.


\paragraph{Benefiter of using DBMS:}
(https://www.tutorialspoint.com/Advantages-of-Database-Management-System)


\begin{itemize}
\item Reducing Data Redundancy
\item  Sharing of Data
\item Data Integrity
\item Data Security
\item Backup and Recovery
\item Data Consistency


\end{itemize}



\paragraph{Disadvantages of using DBMS:}

\begin{itemize}

\item 
\end{itemize}



\section{Rational database}


\paragraph{Rational model:} It is a way to present data by set of tables (relations) that are connected to each other. The data is stored as tuples (rows). The columns of tables  are called attributes.

\paragraph{Schema:} For a rational database the schema is the abstract description of the database. 

























\end{document}